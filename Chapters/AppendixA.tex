
%*******************************************************
% Chapter 1
%*******************************************************

\myChapter{Rudiments of combinatorial analysis}

\begin{refsection}

Combinatorics is a branch of discrete mathematics mainly concerned with existence,
enumeration, classification and optimization of arrangements of
elements  of a given set according to 
prescribed rules, providing powerful tools to count, manage and establish
properties of these
arrangements. 
A review of the very basic  of the subject is mandatory here, since 
even the simplest exercises in probability (in the context of finite sample spaces having equally likely outcomes)
naturally demand for such tools to count in a smart way the number of certain
configurations in the sample space. 
(In fact, from an historical perspective, combinatorics was strongly motivated
by applications in 
probability.)
Much of the material presented here should be already well-known to a broad audience from
undergraduate (or maybe even high-school) courses, nevertheless we collect here
few important techniques  in combinatorial calculus which were used throughout the notes. 
Combinatorial problems can
quickly evolve 
from straightfoward to highly non-trivial and intractable ones and can make
their appearence in unexpected contexts: the final sections are devoted to few
more advanced topics maybe not directly related to probability aimed at
providing a 
flavor of what the whole combinatorics is all about, in order the Reader to acquire a broader
perspective 
on the field of combinatorics, in
particular with an eye to applications in physics. 

Many books were written on combinatorics. 
The main sources for this chapter are:
\begin{itemize}
   \item\textcite[][\S~1]{Comtet:1974}
   \item\textcite[][\S~2]{Feller:1966}
\end{itemize}
for exercises, refer \eg{} to 
\begin{itemize}
   \item\textcite[][\S~1]{Ross:2010}.
\end{itemize}


\section{Factorials and binomial coefficient}
The factorial, binomial and multinomial coefficient of non-negative integer numbers are defined and their basic
properties established. 
They appear quite naturally in many combinatorial problems. 
The reader should be already familiar with these topics from elementary calculus
courses. 
Generalization to non-integer arguments 
is possibile but it is postponed until  \cref{sec:complex_factorial}.


   \subsection{Factorial and binomial coefficient of non-negative integer numbers}
\label{sec:integer_factorial}

   Let $\N$ denotes as usual the set of non-negative integer numbers.
   
   \begin{definition}[factorial]
      The ``factorial'' of 
      $n\in\N$,
      customarily denoted by $\Factorial{n}$%
      \footnote{The modern notation $\Factorial{n}$ was introduce by Christian Kramp
	 in 1808.},
 is defined by  induction by letting
      \begin{enumerate}
	 \item\label{item:factorial1}
	    \begin{math} 
	       \Factorial{0} = 1
	    \end{math};
	 \item\label{item:factorial2}
	    \begin{math}
	       \Factorial{n} = n \Factorial{(n-1)} \condition*{\forall n \in \N
		  \backslash\{0\}}
	    \end{math}.
      \end{enumerate}
   \end{definition}
   Alternatively, the factorial can be defined equivalently by
\begin{enumerate}
	 \item
	    \begin{math} 
	       \Factorial{0} = 1
	    \end{math};
	 \item
	    \begin{math}
      \Factorial{n}= \prod_{k=1}^{n} k \condition*{\forall n \in \N \backslash\{0 \}}
	    \end{math}.
      \end{enumerate}

      Roughly speaking,  both definitions make evident that $\Factorial{n}$
      represents, for $n>0$, the product of the first $n$-th
      non-negative integer numbers:
      \begin{dmath*}
	 \Factorial{n} = \underbrace{1 \cdot \ldots   \cdot n}_{\mathclap{\text{Product
		  of $n$ terms}}}
      \end{dmath*}.

   The position $\Factorial{0} = 1$ reflects the empty product
   convention, which states that the result of multiplying no factors is assumed by convention
   to be equal to 1. 
   It might look rather artificial at this step, but it proves convenient to
   write in a more compact way many recursive formulas involving factorials, 
   (\eg, it simplifies some 
   power series expansion formulas), 
   it allows a combinatorial interpretation, and it is consistent with the
   generalization of factorials to arbitrary real and complex numbers by means of the
   gamma function, to be discussed later in 
      \cref{sec:complex_factorial}.


   \begin{definition}[Binomial coefficient]
      \label{def:binomial}
      Let $n,k\in\N$ be any non-negative integer numbers with $0\leq k \leq n$.
      The ``binomial coefficient'' of $n$ and $k$, 
      traditionally denoted by
      $\binom{n}{k}$%
      \footnote{The modern notation $\binom{n}{k}$ was introduced by A. von
	 Ettinghausen in 1826.},
      is defined by
      \begin{dmath}[label={binom:def}]
	 \binom{n}{k} \coloneqq \frac{\Factorial{n}} { \Factorial{k}
	    \Factorial{(n-k)}} 
      \end{dmath}.
   \end{definition}


   \begin{remark}
      The definition of $\binom{n}{k}$ here is necessarily rectricted to $0\leq k \leq n$
      in order  $n-k$ to be always a non-negative integer, so that
      $\Factorial{(n-k)}$ makes sense using the
      definition made so far. 
      Sometimes, $\binom{n}{k}$ is extended to arbitrary large $k\in\N$ by setting
      \begin{dmath}[label={binom:k>n}]
	 \binom{n}{k} =0 \condition*{\forall k >  n}
      \end{dmath}.
      The range of allowed arguments can be further extend: in
      \cref{sec:complex_factorial} the binomial coefficient  of two arbitrary
      complex numbers will be defined (thus including as special case also the
      binomial of negative integer arguments). 
      We shall see that such extension is consistent with  \cref{eq:binom:k>n}.
   \end{remark}

Among the many identities involving the binomial coefficients, this section
focuses on few basic ones. 

\begin{lemma}[Symmetry]
   For every $n,k\in\N$, 
with $0\leq k \leq n$,
   \begin{dmath}[label={binom:reflection}]
      \binom{n}{k} = \binom{n}{n-k}
   \end{dmath}.
\end{lemma}
\begin{proof}
   \begin{dmath*}[compact]
      \binom{n}{n-k} = \frac{\Factorial{n}} { \Factorial{(n-k)}
	 \Factorial{(n-(n-k))}} = \frac{\Factorial{n}} {\Factorial{(n-k)}
	 \Factorial{k}} = \binom{n}{k}
   \end{dmath*}.
\end{proof}

\begin{lemma}[Pascal's rule]
   For every $n,k\in\N$,
with $0\leq k < n$,
   \begin{dmath}[label={binom:pascal},frame]
      \binom{n+1}{k+1} = \binom{n}{k} + \binom{n}{k+1}
   \end{dmath}.
\end{lemma}

\begin{remark}
   According to \cref{def:binomial}, $\binom{n}{k}$ makes sense for $0\leq
k \leq n$, $\binom{n}{k+1}$ makes sense 
for $0 \leq k+1 \leq n$ and $\binom{n+1}{k+1}$ makes sense for $0 \leq
k +1 \leq n+1$. 
Since we have not defined the binomial coefficient when $k$ takes negative
integer 
values, 
\cref{eq:binom:pascal} does not allow to compute $\binom{n+1}{0}$ since it
would require to set $k=-1$ on the left-hand side,  and this would make
$\binom{n}{-1}$ appearing on the right-hand side. 
Moreover, setting $n=k$ in order to compute $\binom{n+1}{n+1}$ on the left-hand
side would make $\binom{n}{n+1}$ appearing on the right-hand side (this problem
can be circumvent however by using the above convention for $k> n$).
However, both scenarions are easy to treated: $\binom{n+1}{0} =
\binom{n+1}{n+1}  = 1$.  For this reason, we restrict the formula to $0\leq k <
n$.
\end{remark}

\begin{remark}
   Equivalenty, by replacing $k' = k+1$, we have
   \begin{dmath}[label={binom:pascalk}]
      \binom{n+1}{k} = \binom{n}{k-1} + \binom{n}{k}
   \end{dmath}
   with $1 \leq k \leq n$.
\end{remark}

\begin{proof}
  A direct  calculation shows that 
     \begin{dmath*}
	\binom{n}{k} + \binom{n}{k+1 } 
	= 
	\frac{\Factorial{n}}{\Factorial{k}\Factorial{(n-k)}}
	+ 
	\frac{\Factorial{n}}{\Factorial{(k+1)}\Factorial{(n-k-1)}}
	= 
	\frac{\Factorial{n}(k+1)}{\Factorial{(k+1)}\Factorial{(n-k)}}
	+ 
	\frac{\Factorial{n}(n-k)}{\Factorial{(k+1)}\Factorial{(n-k)}}
	= 
	\frac{\Factorial{n}[(k+1) + (n-k)]}{\Factorial{(k+1)}\Factorial{(n-k)}}
	= 
	\frac{\Factorial{n}(n+1)}{\Factorial{(k+1)}\Factorial{(n+1-1-k)}}
	=
	\binom{n+1}{k+1}	
     \end{dmath*}
  \end{proof}
   



\begin{remark}
   \Cref{eq:binom:pascal} gives rise to a nice pictorial representation in terms of
   the so-called Pascal's triangle, see \cref{fig:pascal}.
\end{remark}
   
\begin{figure}
   \centering
   \begin{tikzpicture}
      \foreach \n in {0,...,8} {
	 \foreach \k in {0,...,\n} {
	    \node at (\k-\n/2,-\n) {$ \binomialCoefficient{\n}{\k}$};
	 }
      }
   \end{tikzpicture}
   \caption{Picture of the first eight entries of the  Pascal's
      triangle. The triangle is built recusively in this way: every entry
      represents $\binom{n}{k}$ where $n$ is the row index and $k$ is the
      column index; every entry is evaluated by computing the sum of the
      corresponding two entries $\binom{n-1}{k-1}$ and $\binom{n-1}{k}$ in the
      previous line, following \cref{eq:binom:pascal}.
      The two sides of the triangle are vertically symmetric, as expected from
      \cref{eq:binom:reflection}.
      \label{fig:pascal} }
\end{figure}


\subsection{Newton's formula}

\begin{theorem}
   For every $x,y\in\R$ and $n\in\N$, 
   \begin{dmath}[label={binom:newton}]
      (x+y)^{n} = \sum_{k=0}^{n} \binom{n}{k} x^{k} y^{n-k}
   \end{dmath}
\end{theorem}

\begin{remark}
   Since $x+y = y+x$, \cref{eq:binom:newton} can equivalently be written as
   \begin{dmath*}
      (x+y)^{n} = \sum_{k=0}^{n} \binom{n}{k} x^{n-k} y^{k}
   \end{dmath*}.
   To prove the equivalence with \cref{eq:binom:newton}, replace $m = n-k$, so $k=n-m$ and 
   \begin{dmath*}
      (x+y)^{n} = \sum_{k=0}^{n} \binom{n}{k} x^{k} y^{n-k}
      = \sum_{m=N}^{0} \binom{n}{n-m} x^{n-m} y^{m}
      = \sum_{m=0}^{N} \binom{n}{m} x^{n-m} y^{m}
   \end{dmath*},
   where the symmetry property of the binomial coefficient has been used. 
\end{remark}

\begin{proof}
   By induction on $n\in\N$. For $n=0$, \cref{eq:binom:newton} reads
   \begin{dmath*}
      (x+y)^{0} = \binom{0}{0} x^{0} y^{0}
   \end{dmath*}
   which is true.
   Let us now prove that \cref{eq:binom:newton} for a given $n\in\N$ implies
   \cref{eq:binom:newton} for $n+1$:
   \begin{dmath*}
      (x+y)^{n+1} 
      = 
      (x+y) (x+y)^{n}
      = 
      (x+y)\sum_{k=0}^{n} \binom{n}{k}x^{k} y^{n-k}
      = 
      \sum_{k=0}^{n} \binom{n}{k}x^{k+1} y^{n-k}
      +
      \sum_{k=0}^{n} \binom{n}{k}x^{k} y^{n+1-k}
      = 
      x^{n+1} 
      +
      \sum_{k=0}^{n-1} \binom{n}{k}x^{k+1} y^{n-k}
      +
      y^{n+1}
      +
      \sum_{k=1}^{n} \binom{n}{k}x^{k} y^{n+1-k}
      = 
      x^{n+1} 
      +
      \sum_{k=1}^{n} \binom{n}{k-1}x^{k} y^{n+1-k}
      +
      y^{n+1}
      +
      \sum_{k=1}^{n} \binom{n}{k}x^{k} y^{n+1-k}
      =
      x^{n+1} 
      +
      y^{n+1}
      +
      \sum_{k=1}^{n} \left[ \binom{n}{k-1} + \binom{n}{k} \right] x^{k} y^{n+1-k}
      =
      x^{n+1} 
      +
      y^{n+1}
      +
      \sum_{k=1}^{n} \binom{n+1}{k} x^{k} y^{n+1-k}
      =
      \sum_{k=0}^{n+1} \binom{n+1}{k} x^{k} y^{n+1-k}
   \end{dmath*}.
\end{proof}

  Implications: 
  \begin{dgroup*}
     \begin{dmath*}[compact]
	\sum_{k=0}^{n} \binom{n}{k} = (1+1)^{n} = 2^{n}
     \end{dmath*},
     \begin{dmath*}[compact]
	\sum_{k=0}^{n} (-1)^{k} \binom{n}{k} = (1-1)^{n} = 0
     \end{dmath*}.
  \end{dgroup*}

\begin{advanced}

   \subsection{Vandermond's identity}

   The following identity, which goes under the name of Vandermonde's identity,
   is relevant for discussing the hypergeometric probability distribution
   (\cref{sec:hypergeometric_distribution}); in particular, the normalization
   condition of the hypergeometric distribution follows directly from  the Vandermond's
   identity.
   An algebraic proof of this identity is straightforward and relies on the Cauchy product of two
   polynomials.

   \begin{lemma}[Vandermond's identity]
      For every $m,n,l\in\N$, 
      \begin{dmath}
	 \binom{n+m}{k} = \sum_{l=0}^{k} \binom{m}{k}\binom{n}{k-l}
      \end{dmath}.
   \end{lemma}

   \begin{proof}

      Consider
      \begin{dmath*}
	 (1 + x)^{m+n} = (1+x)^{m} (1+x)^{n}
      \end{dmath*}.
      Both sides can be expanded using  Newton's law:
      \begin{dmath*}
	 \sum_{l=0}^{n+m} \binom{n+m}{l} x^{l} = \left( \sum_{i=0}^{m}
	    \binom{n}{i}x^{i}\right) \left( \sum_{j=0}^{n} \binom{n}{j} x^{j}\right)
	 = 
	 \sum_{l=0}^{m+n} \left( \sum_{j=0}^{i}
	    \binom{m}{i}\binom{n}{j-i}\right) x^{i}
      \end{dmath*}.
      Applying the Cauchy's product formula on the right-hand side we get
   \end{proof}

  \subsection{Multinomial coefficient of non-negative integer arguments}

  \begin{definition}
     Let $n,m\in \N$.
     For every $m$-dimensional configuration
     \begin{dmath*}
	\begin{pmatrix} n_{1} \\ \vdots \\ n_{m} \end{pmatrix} \in \N^{m}
     \end{dmath*},
     such that $0 \leq n_{k} \leq n $ for all $0\leq k \leq m$ and 
     satisfying the constrain 
     \begin{dmath}[label={multinomial:sumn}]
	\sum_{k=1}^{m} n_{k} = n
     \end{dmath},
     the corresponding ``multinomial'' coefficient 
$\binom{n}{n_{1}, n_{2}, \ldots, n_{m}} $
     is defined as 
     \begin{dmath}[label={multinomial:def1},frame]
	\binom{n}{n_{1}, n_{2}, \ldots, n_{m}} \coloneqq
	\Factorial{n} 
	\prod_{k=1}^{m}\frac{1}{\Factorial{n_{k}}} 
     \end{dmath}.
  \end{definition}
  Sometimes, the same multinomial coefficient is also denotes as follow
  \begin{dmath}[label={multinomial:def2},frame]
	\Multinomial{n_{1}, n_{2}, \ldots, n_{r}} \coloneqq
	\frac{\Factorial{\left( \sum_{k=1}^{r}n_{k} +
	      \right)}}{\prod_{k=1}^{r}\Factorial{n_{k}}} 
     \end{dmath},
  where the number $n$ is not explicitly written but can be easily recovered using the constrain
  \cref{eq:multinomial:sumn}.

  The binomial coefficient is a special case of the multinomial coefficient
  using $m=2$, $n_{1} =k$, $n_{2} = n -k$, $0\leq k \leq n$.


   \section{A number of additional properties of factorials and binomial
      coefficients}
   \subsection{Factorial and binomial coefficient of non-integer numbers}
\label{sec:complex_factorial}

Factorials and binomial coefficients can be generalized to arbitrary complex
arguments in terms of the higher-trascendental gamma function%
\footcite[For a review of properties, in particular about the case of negative
integer arguments, have a look at \eg][]{Onl-kron:2015}.

There is a number of equivalent ways to define the gamma special function. 
One way to define it is as the \emph{unique} extension (up to certain
regularity assumptions) of the factorial to real and complex numbers.
The precise statement is provided by the theorems of Bohr-Mollerup (for
the real case) and Wielandt (for the complex case).
This approch makes clear how gamma function is related to the factorial.
However, it should be stressed that the usefulness of gamma function cannot be
understimated, it goes well beyond  this and finds 
many applications also in contexts where the link with factorials is less evident. 


\subsection{Binomial inversion}


\subsection{Pascal matrices}

\subsection{Relation  with Fibonacci numbers}

Fibonacci numbers are defined in 
   \cref{sec:fibonacci}. 
   There exists a relation between binomial coefficients and Fibonacci numbers.
\begin{lemma}
   For every $n\in\N\backslash\{ 0 \}$, 
   \begin{dmath}
      \Fibonacci{n} = \sum_{k=0}^{\floor*{\frac{n-1}{2}}} \binom{n-k-1}{k}
   \end{dmath},
   where $\floor{\cdot}$ denotes the floor function%
   \footnote{The floor function $\fullfunction{\floor{\cdot}}{\R}{\N}$ is
      defined as $\floor{x} = \sup[n\in\N]{\Set{n| x \geq n }}$. The notation
      is due to Kenneth E. Iverson.},
   and $\Fibonacci{n}$ denotes the
   \mynth{$n$} Fibonacci number.
\end{lemma}



\end{advanced}
   \section{Basic set theory and the power set}

   \begin{definition}
      Let $\Omega$ be any set.
      The set of all subsets of $\Omega$ is called the ``power set'' of
      $\Omega$ and is denoted by $2^{\Omega}$.
   \end{definition}

   The existence of the power set is an axiom in standard set theory.
   For every $\Omega$, the empty set and $\Omega$ itself are in $2^{\Omega}$:
   $\varnothing \in 2^{\Omega}$ and 
   $\Omega \in 2^{\Omega}$. 
   The special notation $2^{\Omega}$ will become clear later when discussing
   the ``cardinality'' of $2^{\Omega}$ for finite sets $\Omega$.

   
   \begin{lemma}
Let $\Omega$ a finite set with cardinality $N\in\N$.
Then, $2^{\Omega}$ has cardinality $2^{N}$.
   \end{lemma}

   In words: if $\Omega$ has $N$ and only $N$ distinct elements, then there are 
   $2^{N}$ possible distinct subsets of $\Omega$ (including $\Omega$ itself and
   the empty set).

   \section{Permutations}

   \subsection{Permutations without repetition}

   Intuitively speaking, a ``permutation'' of a set of $N$ elements
   should be thought as a way of sorting or re-arranging the $N$ elements.
   At a mathematical level, this is implemented as a bijection from the set of
   $N$ elements to itself.
   
   \begin{definition}[Permutation]
      Let $\Omega$ be a set.
      A ``permutation'' of $\Omega$ is any bijective application fro $\Omega$
      to itself. 
   \end{definition}

   \begin{lemma}
      Let $\Omega$ be a finite set with $\card{\Omega} = N$ for some $N\in\N$.
      The number of all and only the permutations of $\Omega$ is
      $\Factorial{n}$.
   \end{lemma}
   \begin{proof}


   \end{proof}

   \subsection{Permutations with repetition}
   

   \section{Dispositions} 
   \subsection{Dispositions without repetition}
   \subsection{Dispositions with repetition}
   
   \section{Combinations}
   \subsection{Combinations without repetition}
   

   \begin{definition}[combinations]

      Let $\Omega$ a finite set of cardinality $\card{\Omega} = N$ for 
      $N\in\N$ and 
      $k\in\N$, with  $0\leq k \leq N$.
      A ``$k$-combination'' of $A$ is any subset of $\Omega$ of
      cardinality $k$.
   \end{definition}

   In words: a $k$-combination of a set $\Omega$ of $N$ elements  is any subset
   of $\Omega$ having precisely $k$ distinc elements. 

   In combinatorics we refer to a $k$-combination of $\Omega$ as a ``way of 
   extracting $k$ different elements out of a set of   $N$ elements'', or as a 
   ``way of extracting $k$ elements at a time from a set of $N$ elements''.
    
   If we think of a $k$-combination as a way to ``extract'' $k$ different
   elements from a set of $N$ total elements, it should be stressed that 
   \begin{itemize}
	 \item there is no ``repetition'': all $k$ elements are distinct ones; (in
	    fact, a $k$-combination is just a subset of $\Omega$, so we can't
	    have the same element of $\Omega$ more than once in a given
	    $k$-combination);
	 \item there is no ``order'': given a $k$-combination $A$ of $\Omega$,
	    we can only say whether a given element $\omega \in \Omega$ belongs
	    to $A$ or not; there is no notion that allows us to ``sort'' the
	    elements of $A$.
      \end{itemize}

   To introduce repetion we should work with $\Omega^{k}$ instead of $\Omega$;
   this will lead us to discuss ``$k$-combinations with repetition''.
   To introduce a notion of ``order'' in $\Omega$ or $\Omega^{k}$ we will
   consider applications from $\Set{1,\ldots,k}$ to $\Omega$: this will lead us
   to discuss dispositions (arrangements).

   \begin{lemma}
      Let $\Omega$ a finite set of cardinality $\card{\Omega} = N$ for 
      $N\in\N$ and 
      $k\in\N$, with  $0\leq k \leq N$.
      The subset of $2^{\Omega}$ of all and only the $k$-combinations of
      $\Omega$ has cardinality $\binom{N}{k}$.
   \end{lemma}

   In words: the number of all and only the possible subsets of $\Omega$ having
   precisely $k$ many items is $\binom{N}{k}$.

   \begin{proof}
   \end{proof}

   \subsection{Combinations with repetition}

\begin{table}
  \caption{Summary}
  \begin{tabularx}{\textwidth}{Xll} \toprule
    & \tableheadline{without repetition} & \tableheadline{with repetition} \\ \midrule
    permutations of $n$ elements & $\Factorial{n}$ & $\Factorial{N} \prod_{j=1}^{k}  \frac{1}{\Factorial{n_{j}}}$ \\
    $k$-combinations of $n$ elements & $\binom{n}{k}$ & $\binom{n+k-1}{k}$ \\
    $k$-arrangements of $n$ elements & $\dfrac{\Factorial{n}}{\Factorial{(n-k)}}$ & $n^{k}$ \\
    \bottomrule
  \end{tabularx}
\end{table}



   \section{Inclusion-exclusion theorem}

   \begin{figure}
      \centering
      \begin{tikzpicture}
	 \begin{scope}[shift={(3cm,-5cm)}, fill opacity=0.5]
	    \fill[red] \firstcircle;
	    \fill[green] \secondcircle;
	    \fill[blue] \thirdcircle;
	    \draw \firstcircle node[below] {$A$};
	    \draw \secondcircle node [above] {$B$};
	    \draw \thirdcircle node [below] {$C$};
	 \end{scope}
      \end{tikzpicture}
      \caption{Illustration of the inclusion-exclusion strategy}
   \end{figure}

   \section{What's the use of all this?}
   Combinatorics prove useful 


\printbibliography[heading=subbibliography]
\end{refsection}
   


