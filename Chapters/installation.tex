
%*******************************************************
% HDF5 chapter
%*******************************************************

\myChapter{Installing Android SDK}

This appendix goes through the steps needed to install and configure
properly all
the appropriate tools to setup an Android development environment and
to get started developing Android applications%
\footcite[Refer for example to][\S~1 for further
details.]{Mednieks.Dornin.ea:2012}.

\section{Installation of the tools}

To setup a suitable framework for developing Android applications, the following are
needed:
\begin{itemize}
   \item latest Java Development Kit (JDK)
   \item latest Eclipse Integrated Development Environment (IDE)
   \item latest Android SDK
   \item latest Android Development Tools (ADT) Plu-in for Eclipse
\end{itemize}

The use of Eclipse is not strictly necessary of course, but it is strongly
suggested, since it greatly simplifies
writing, organizing, compiling, maintaining and debugging the soource code, and
there is an active community to consult online%
\autocite[Regarding the use of Eclipse for developing Android Apps,
see in particular][]{Cinar:2012}
The JDK and SDK tools are mandatory.

\subsection{Installing JDK\label{sec:JDK}}

First of all, we need to install the Java Development Kit (JDK).
This can be freely downloaded from the Oracle's website%
%\autocite{Onl-orac:2014}.
\footnote{Visit
   \url{http://www.oracle.com/technetwork/java/javase/downloads/index.html}}.
There are automatic installers for \macos{} and \windows{} platform.
On \linux{}, download the archive, extract the JDK folder, then some
additional configuration is needed to make the JDK available and properly
configured, please follow instructions in the install file.


\subsection{Installing Eclipse IDE\label{sec:Eclipse}}

Download the most recent version of Eclipse, available through
the Eclipse's website%
%\autocite{Onl-ecli:2014}.
\footnote{Visit \url{http://www.eclipse.org/downloads/}}.  
Eclipse comes with several packages available.
A possible choice it to select 
the \emph{Java EE Developer package} (the ``Enterprise
Edition''), although the Eclipse classic or the Eclipse standard Developer
package will work as well.

Once you've decompressed the downloaded archive, you'll have a folder called eclipse.
Simply drag this entire folder to the Applications folder.
You can run the Eclipse IDE by clicking on the application icon inside the
folder.

\subsection{Installing Android SDK\label{sec:SDK}}

The Android SDK is a collection of files. 
There is no installation script.
To install the SDK, download the SDK package that corresponds to your system%
\footnote{Visit
   \url{http://developer.android.com/sdk/installing/index.html?pkg=tools}}.
The download is an archive.
Open the archive and extract the folder at the top level of the archive where
you want to install it, for example in your home folder.


\subsubsection{The SDK manager\label{sec:SDK_manager}}

The SDK manager allows you to install packages in the SDK that will support
multiple versions of the Android OS and multiple API levels.
For example, when the SDK Tools are updated or a new version of the Android
platform is released, you can use the SDK Manager to quickly download them to
your environment.

In \cref{sec:ADT} we will install the ADT plugin for Eclipse, and after that
you will be able to launch the SDK manager also via Eclipse.
For the time being, assuming that you have not installed the AVT plugin yet, we
will run the SDK Manager directly from its installation folder. 

\begin{remark}
   For easy access to the SDK tools from a command line, add the location
   of the SDK's tools/ and platform-tools to your PATH environment variable.
\end{remark}

Here is an outline of the packages required and some which are recommended by
the Android website:
\begin{description}
   \item[SDK Tools]
      Required. 
      Your new SDK installation already has the latest version. Make sure you
      keep this up to date.
   \item[SDK Platform-tools]
      Required. You must install this package when you install  the SDK for the
      first time. 
   \item[SDK Platform]
      Required.
      You must download at least \emph{one platform} into your environment so
      you are able to compile your application.
      In order to provide the best user experience on the latest devices, we
      recommend that you use the latest platform version as your build target.
      You'll still be able to run your app on older versions, but you must
      build against the latest version in order to use new features when
      running on devices with the latest version of Android.
   \item[System image]
      Recommended.
      It's a good practice to download system images for all versions of
      Android that your app will support and test your app running on
      them using the Android emulator.
   \item[Android support]
      Recommended.
      It is not mandatory, however please consider that all of the activity
      templates available when creating a new project with the ADT Plugin
      (\cref{sec:ADT}) require this.
\end{description}

The above list is not comprehensive and you can add new sites to download
additional packages from third-parties.

\subsection{Setting ADT plugin for Eclipse\label{sec:ADT}}

\subsubsection{Installing the ADT}

We will use the Install New Software Wizard to download and install the ADT
plugin in Eclipse%
\footnote{The most recent documentation on how to install the ADT can be found
   at \url{http://developer.android.com/sdk/installing/installing-adt.html}}.
Proceed through the following steps:
\begin{enumerate}
   \item Launch Eclipse on your computer
   \item Start Eclipse, then select Help > Install New Software.
    \item Click Add button, in the top-right corner.  In the Add Repository dialog that
      appears, enter "ADT Plugin" for the Name and the following URL for the
      Location: \url{https://dl-ssl.google.com/android/eclipse/}.
      Note: The Android
      Developer Tools update site requires a secure connection. Make sure the
      update site URL you enter starts with HTTPS.
   \item Click OK button
   \item In the Available Software dialog, select the checkbox next to
      Developer Tools and click Next.
   \item In the next window, you'll see a list of the tools to be downloaded.
      Click Next.
   \item Read and accept the license agreements, then click Finish.
      If you get a security warning saying that the authenticity or validity of
      the software can't be established, click OK.
   \item When the installation completes, restart Eclipse.
\end{enumerate}

\begin{figure}
   \centering
   \includegraphics[width=.7\textwidth]{./Images/ADT_installer.png}
   \caption{Screenshot of using the Install New Software Wizard to download and
      install the ADT plugin in Eclipse}
\end{figure}

\subsubsection{Configuring the ADT}

\section{Install additional tools}

In this section we discuss some additional tools and alternatives which can be helpful to work
with the Android emulator, namely
\begin{itemize}
   \item \intelHAXM{} driver
   \item \genymotion{} Android emulator
\end{itemize}

\subsection{\intelHAXM{}\label{sec:HAXM}}

This section will guide through installing the Intel\copyright{} Hardware Accelerated
eXecution Manager (\intelHAXM), a hardware-assisted virtualization engine
(hypervisor) that uses Intel\copyright{} Virtualization Technology (VT) to speed up
AVD emulator performance.
The instructions and information in this section are taken from the official Intel\copyright{} webpage%
\footnote{Check the following URL and subpages: %Visit 
  % \url{https://software.intel.com/en-us/android/articles/installation-instructions-for-intel-hardware-accelerated-execution-manager-mac-os-x}}.
  \url{https://software.intel.com/en-us/android/articles/intel-hardware-accelerated-execution-manager}}.
Check this page for updated information.

The \intelHAXM{} is freely available for download.
\intelHAXM{}
requires that your development environment runs on Intel
hardware, in particular the following requirements are needed:
\begin{itemize}
   \item Hardware requirements:
      \begin{itemize}
   \item Intel\copyright{} processor with support for Intel\copyright{} VT-x,
      Intel\copyright{} EM64T (Intel\copyright{}
      64), and Execute Disable (XD) Bit functionality
   \item At least 1 GB of available RAM
\end{itemize}
\item Supported operating systems:
   \begin{itemize}
   \item Mac OS X: Mac OS X 10.6 (32/64-bit) or 10.7 (32/64-bit) or 10.8 (64-bit) or
      10.9 (64-bit)
   \item Microsoft Windows: Windows 8 and 8.1 (32/64-bit), Windows 7
      (32/64-bit), Windows Vista* (32/64-bit) 
\end{itemize}
\end{itemize}
As prerequisite, Android SDK (version 17 or higher) should be
available in your developer environment (we have installed it in \cref{sec:SDK}).




To uninstall Intel HAXM on a Mac, open a terminal window and execute this
command:
\begin{lstlisting}
	$ sudo /System/Library/Extensions/intelhaxm.kext/Contents/Resources/uninstall.sh
\end{lstlisting}
You will be prompted for your current user password. Follow the uninstaller
prompts to remove \intelHAXM{}.

Warning: Close all instances of the Android x86 emulator before removing
\intelHAXM{}. \intelHAXM{} cannot be removed while in use.

\subsection{\genymotion{}\label{sec:genymotion}}

Genymotion is an emulator using x86 architecture virtualization, which makes it much
more efficient. 
Taking advantage of OpenGL hardware acceleration, it allows you to test your
applications with amazing 3D performance.
The performance of the \genymotion{} emulator makes it  particularly well-suited
for everyday support while developing
new functionalities and when you would like to check quickly how the new work
done looks on
an the emulator.
Furthermore,
it offers interesting command line tools to build complex tests.

\genymotion{} is available for 
Windows, Mac and Linux. The base version is free.
\genymotion{} can be easily integrated within the Eclipse IDE.

For a full list of features and installation requirements, see the official
webpage%
\footnote{Visit \url{https://cloud.genymotion.com/page/doc/}}.

\subsubsection{\genymotion{} Installation}

\genymotion{} requires Oracle VirtualBox >= 4.2.12 (greater is better) to be
installed. 

To install \genymotion{} you need to 
\begin{itemize}
   \item Create an account on \genymotion{} website, if you don't have already
      one.
   \item Download and install Oracle VirtualBox, if you don't have it already
      installed in your computer.
   \item Download \genymotion{} from the official website.
   \item Follow the instructions in the installer.
\end{itemize}

After lauching \genymotion{}, you should login with an existing \genymotion{}.
Once connected, you will see all the available virtual devices, and you will be
able to select the desired device and to run it.

\subsubsection{Installing \genymotion{} Eclipse plugin}

The plugin can be installed using the Install New Software Wizard, in much the
same way we have followed for installing the AVT plugin (\cref{sec:AVT}). Of course, we need to change the
source URL when Eclipse should look for the plugin. The details steps are as
follows:
 \begin{enumerate}
   \item Launch Eclipse on your computer
   \item Start Eclipse, then select Help > Install New Software.
    \item Click Add button, in the top-right corner.  In the Add Repository dialog that
       appears, enter any name (\eg, ``\genymotion{}'') for the Name field and the following URL for the
      Location field: \url{http://plugins.genymotion.com/eclipse}
   \item Click OK button
   \item In the Available Software dialog, select the checkbox next to
      Developer Tools and click Next.
   \item In the next window, you'll see a list of the tools to be downloaded.
      Click Next.
   \item Read and accept the license agreements, then click Finish.
      If you get a security warning saying that the authenticity or validity of
      the software can't be established, click OK.
   \item When the installation completes, restart Eclipse.
\end{enumerate}

     
\section{Configuring and running the AVD\label{sec:AVD}}

\section{Keeping up-to-date\label{sec:keeping_updated}}
