
%*******************************************************
% Chapter 1
%*******************************************************

\myChapter{Introduction}

This chapter gives an overview of the subject and discuss 
the roadmap of the next chapters.

\section{Top 10 algorithms in data mining}
In an effort to identify some of the most influential algorithms in data
mining, 
the IEEE International Conference on Data Mining
identified the following top 10 algorithms:
\footcite[See][]{Wu.Kumar.ea:2007}
\begin{itemize}
   \item C4.5 (decision trees)
   \item The $k$-means algorith;
   \item Support vector machines (SVM);
   \item The Apriori algorithm;
   \item The EM algorithm;
   \item PageRank;
   \item AdaBoost;
   \item $k$-Nearest Neighbor classification;
   \item Na\"{\i}ve Bayes;
   \item CART.
\end{itemize}

\section{Goals}

These notes are aimed at developing a throughly understanding of the Top 10
algorithms un data mining, from both theoretical and practical perspective.

It is intrinsically cross-field subject, requiring knowledge from different
fields (statistics, mathematics, etc)
Moreover, it requires to grasp computer science at several different scales.
Require widespread knowledge  of several different fields.
\begin{itemize}
\item Programming languages: different companies uses different programming
   languages and/or software tools; some languages are useful because they
   allow less developing time, some other are more suitable for high
   performance computing, some other have powerful and highly maintened
   libraries, some other are the ones employed in large scale distributed
   platforms. So you need to somewhat familiar and confident with several
   programming language, to choose the one most suitable for your task or to
   integrate 
\item Feature extraction
\item Data cleaning
\item Knowledge of non trivial statistics  needed in the background
\item Mathematics involved in these algorithms is not trivial, but you need a
   deep knowledge of the algorithms in order to understand 1) where they are
   epected to be the best option to approach the problem, 2) how to improve the
   algorithm for the problem at hand, 3) to gain better understanding of the
   results of the algorithm (to catch artifacts, etc)
\item Database
\item Visualization
\item Debuggin
\item Intuition
\end{itemize}

There is (and there can be) no single source covering well all these aspects of
machine learning. Of course, there would be not enough room in a single volume to explore
in details all these things. However, here we will try to procee d in a way to
offer at the reader 1) knowledge of the key algorithms, 2) implementing those
algorithms in full details in some programming language to attack simple
problems, 3) and try to make a full analysis of real problems to show some
examples, trying to
fill the gap between theory and practice. It goes without saying that some
choices will be a matter of taste. For example, we will need to choose a
programming language. We will discuss to some extend pro and cons of different
programming languages.

Out approach will be
\begin{itemize}
   \item Python: best tool to quick explore the data, data visualization etc
   \item \cpp{} to develop heavy programs
\end{itemize}

We will also give a quick account of some other tools, for example
\begin{itemize}
   \item TMVA: Root Cern machine learning framework used in High energy physics
   \item Tools in R
\end{itemize}
but we will not use them extensively in the applications.

Regarding the use of Java, this is mostly related to Hadoop and related
Map-Reduced implementations. 



%\myChapter{TVMA}
%\myChapter{HDF5}
%\myChapter{MLPACK}

