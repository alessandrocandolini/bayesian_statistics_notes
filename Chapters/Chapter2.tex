
%*******************************************************
% Chapter 1
%*******************************************************

\myChapter{Random variables}
\label{chap:random_variables}

\begin{refsection}

   Discrete and continuous, univariate and multivariate real random variables are
   discussed, together with a bunch of practical tools helping to handle with
   them, specifically: probability
   density function, cumulative distribution function, moments
   (\ie, mean, variance, skewness, kurtosis, etc).
   All of this provides a more concrete, visualizable approach to probability
   built on top of the
   construction of the previous chapter by means of probability measures; it should be stressed however
   that the probability measures provides a more general framework. 
   This chapter also covers functions of random variables, change of variables
   (\ie, what physicists are accustomed to call ``propagation of errors''), etc.
   Generating functions --- another useful tool to deal with probability
   distributions --- are postponed to \cref{chapt:generating_functions}: an
   entire chapter is devoted to them 
   since their theory and their application deserve special attention. 
   More general  random variables (\eg, matrix-valued random variables)
   although useful are not presented here (see however
   \cref{chap:random_matrices}, which is 
   devoted to a brief account of random matrix
   theory).

   \section{Definitions and basic properties}

      Let 
      \begin{inparaenum}[(a)]
      \item $(\Omega, \Sigma, \P{\cdot})$ be any probability space;
	 \item $(E, \epsilon)$ be any measure space;
	 \item $\fullfunction{X}{\Sigma}{E}$ be any (measurable) function from
	    $\Sigma$ to $E$.
      \end{inparaenum}
      In this context: $\Omega$ is called the ``sample space'', $E$ is called ``state space'',
      $X$ is called a ``random variable'', ``aleatory variable'' or ``stochastic variable''.

      Whenever the range $X(\Sigma) \subseteq E$ of $X$ has finite or countably
      infinite cardinality, the
      random variable $X$ is called a ``discrete'' random variable.
      When the range of $X$ is uncountably infinite instead, $X$ is called a
      ``continous'' random variable. 

      In this chapter, we will be mainly interested in the  case of \emph{real}-valued
      (discrete or continuous) univariate (\ie, $E \subset \R$) or multivariate
      (\ie, $E\subseteq \R^{N}$, $N\in\N$, $N\geq 2$) random variables (the
      Borel algebra being the underlying $\sigma$-algebra in the state space).
      For such case, it makes sense and it proves useful to introduce the
      important concepts of moments, distribution functions, etc. (Those definitions are
      slightly different whether the random variable is discrete or continuous
      and univariate or multivariate,
      as we will see shortly.)
      However, the definition of random variables applies to more general
      settings; for example, in \cref{chap:random_matrices} the case where $E$
      is a suitable space  of matrices 
      will be taken into account.
      Other scenarios are also possible (\eg, random graphs, etc) some being
      relevant for statistical mechanics, machine learning, etc (see later chapters).


   It is worth notice that random variables take value over the set $E$,
   nevertheless $E$ is not the probability space itself; the random variable
   instead is defined as a mapping defined over a suitable $\sigma$-algebra
   of an underlying probability
   space $\Omega$ and targetting the (measure) state space $E$. 

   What is the benefit of having defined a random variable as a mapping from a
   probability space to the state space $E$ instead of having $E$ itself
   equipped with a suitable probability
   measure?

   \section{Probability distribution}

   In most applications, the underlying probability space defining a random
   variable remains hidden.
   Instead, a ``distribution function'' is introduced which 



   \subsection{Univariate discrete random  variable}
   \subsection{Univariate continuous random variable}
   \subsection{Multivariate discrete random variable}
   \subsection{Multivariate continuous random variable}

   \section{Moments of a distribution}
   \section{Transformations of random variables}
   \section{Buffon's needle}

   \section{What's  the use of all this?}

\printbibliography[heading=subbibliography]
\end{refsection}
