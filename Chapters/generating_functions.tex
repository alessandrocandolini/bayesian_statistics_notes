
%*******************************************************
% Chapter 1
%*******************************************************

\myChapter{Generating functions}
\label{chapt:generating_functions}
\begin{refsection}
Generating functions are a overwealming  weapon to attack irresistable problems
which fail a 
more straightforward approach. 
In this section we will introduce the notion of generating function, we will
establish the relevant properties and we apply the technique of generating
functions to few prototypical examples, in particular involving the famous
Fibonacci numbers and the integer partitions. 
In later chapters, the technique of generating functions will prove useful for
example to deal with Galton-Walton problem.
In statistical mechanics, the generating function approach is related to the
grancanonical partition function.
The main source for this chapter is \cite{Wilf:1994}.


   \section{Fibonacci numbers as a prototype}
   \label{sec:fibonacci}
   The  sequence $(\Fibonacci{n})_{n\in\N}$  of Fibonacci numbers can be defined in a number of equivalent ways and
historically made its first appearance in connection with a combinatorial
model (the rabbit's population model described below).
One definition is the following: the sequence $(\Fibonacci{n})_{n\in\N}$ of
Fibonacci numbers is defined as 
the \emph{unique} solution of 
\begin{dmath}[label={Fn}]
   F_{n+2} = F_{n} + F_{n+1} \condition*{n\in\N}
\end{dmath}
satisfying the initial conditions
\begin{dseries}
   \begin{math}
      F_{0} = 1
   \end{math},
   \begin{math}
      F_{1} = 1
   \end{math}.
\end{dseries}
\Cref{eq:Fn} is a second-order homogeneous linear recursive equation with
constant coefficients, the general theory of such equations ensure that there
exists one and only one solution of it satisfying the initial valued problem.






   \section{Definition and properties of generating functions}

   \section{Solution of Fibonacci recurrence using generating functions}

   

\printbibliography[heading=subbibliography]

\end{refsection}
